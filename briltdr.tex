% Customizable fields and text areas start with % >> below.
% Lines starting with the comment character (%) are normally removed before release outside the collaboration, but not those comments ending lines

%%%%%%%%%%%%% local definitions %%%%%%%%%%%%%%%%%%%%%


%%%%%%%%%%%%%%%  Title page %%%%%%%%%%%%%%%%%%%%%%%%
%\cmsNoteHeader{XXX-08-000}
%%%%%%%%%%%%%%%%%%%%%%%%%%%%%
% This is over-written in the CMS environment: useful as preprint no. for export versions
% >> Title: please make sure that the non-TeX equivalent is in PDFTitle below for papers. For PASs, PDFTitle can be used with plain TeX.
\title{The Phase-2 Upgrade of the CMS Beam Radiation, Instrumentation, and Luminosity Detectors: Technical Design Report}

% >> Authors
%Author is always "The CMS Collaboration" for PAS and papers, so author, etc, below will be ignored in those cases
%For multiple affiliations, create an address entry for the combination
%To mark authors as primary, use the \author* form
%%%%%%%%%%%%%%%%%%%%%%%%%%%%%%%%%%%

%\address[cern]{CERN}
%\author*[cern]{A. Cern Person}
%%%%%%%%%%%%%%%%%%%%%%%%%%%%%%%%%%
% >> Date
% The date is in yyyy/mm/dd format. Today has been
% redefined to match, but if the date needs to be fixed, please write it in this fashion.
\date{\today}

% >> Abstract
% Abstract processing:
% 1. **DO NOT use \include or \input** to include the abstract: our abstract extractor will not search through other files than this one.
% 2. **DO NOT use %**                  to comment out sections of the abstract: the extractor will still grab those lines (and they won't be comments any longer!).
% 3. For PASs: **DO NOT use CMS tex macros.**...in the abstract: CDS MathJax processor used on the abstract doesn't understand them _and_ will only look within $$. The abstracts for papers are hand formatted so macros are okay.
\abstract{
   Your abstract here.
}

% >> PDF Metadata
% Do not comment out the following hypersetup lines (metadata). They will disappear in NODRAFT mode and are needed by CDS.
% Also: make sure that the values of the metadata items are sensible and are in plain text with the possible exception of the PDFtitle for a PAS. Then you can use pure TeX symbols as if on a typewriter. Examples: $\sqrt{s}=13\TeV$ => $sqrt{s}=$ 13 TeV; 32\fbinv => 32 fb$^{-1}$
% No unescaped comment % characters.
% No curly braces {} except for TeX in the PDFtitle.
\hypersetup{%
pdfauthor={G. Auzinger},%
pdftitle={The Phase-2 Upgrade of the CMS Beam Radiation, Instrumentation, and Luminosity Detectors: Technical Design Report},%
pdfsubject={CMS},%
pdfkeywords={CMS, bril, Phase-2 upgrade}}

\newlength\cmsTabSkip\setlength{\cmsTabSkip}{1ex}

\newcommand{\mmsq}{\unit{mm$^2$}}
\newcommand{\mmcu}{\unit{mm$^3$}}
\newcommand{\cmsq}{\unit{cm$^2$}}
\newcommand{\cmcu}{\unit{cm$^3$}}
\newcommand{\percmsq}{\unit{cm$^{-2}$}}
\newcommand{\sigmavis}{\ensuremath{\sigma_{\text{vis}}}\xspace}
\newcommand{\betastar}{\ensuremath{\beta^{*}}\xspace}
\newcommand{\ccdeff}{\ensuremath{\text{CCD}_{\text{eff}}}\xspace}
\newcommand{\hzub}{\ensuremath{\text{Hz}/\mu\text{b}}\xspace}
\newcommand{\utca}{\ensuremath{\mu\mathrm{TCA}}\xspace}
\newcommand{\percmsns}{\ensuremath{\text{cm}^{-2}\,\text{s}^{-1}}}

\newcommand{\IT}{Inner Tracker\xspace}
\newcommand{\OT}{Outer Tracker\xspace}

\newcommand{\TBPX}{Tracker Barrel Pixel Detector\xspace}
\newcommand{\TFPX}{Tracker Forward Pixel Detector\xspace}
\newcommand{\TEPX}{Tracker Endcap Pixel Detector\xspace}

\newcommand{\draft}[1]{{\color{blue}#1}}
\newcommand{\editors}[1]{{\color{yellow}Editors: #1}\\}
\newcommand{\phasezero}{Phase-0\xspace}
\newcommand{\phaseone}{Phase-1\xspace}
\newcommand{\PhaseOne}{Phase-1\xspace}
\newcommand{\phasetwo}{Phase-2\xspace}
\newcommand{\PhaseTwo}{Phase-2\xspace}
\newcommand{\runone}{Run 1\xspace}
\newcommand{\runtwo}{Run 2\xspace}
\newcommand{\runthree}{Run 3\xspace}
\newcommand{\runfour}{Run 4\xspace}
\newcommand{\cmsTable}[1]{\resizebox{\textwidth}{!}{#1}}

\newcommand{\nts}[1]{(\textcolor{red}{Note To Self: #1})}
\newcommand{\nte}[1]{(\textcolor{blue}{Subeditor: #1})}
\newcommand{\nta}[1]{(\textcolor{blue}{Note To Anne: #1})}
\newcommand{\ntd}[1]{(\textcolor{green}{Note To David: #1})}
\newcommand{\ntso}[1]{(\textcolor{green}{Note To Sophie Igor: #1})}
\newcommand{\ntv}[1]{(\textcolor{green}{Note To Vitalii: #1})}
\newcommand{\ntav}[1]{(\textcolor{blue}{Note To Arkady: #1})}
\newcommand{\ntcms}[1]{(\textcolor{magenta}{Note To CMS: #1})}
\newcommand{\ntlhc}[1]{(\textcolor{brown}{Note To LHC: #1})}
\newcommand{\ntp}[1]{(\textcolor{blue}{Note To Paul please check as language editor #1})}
\newcommand{\ntg}[1]{(\textcolor{blue}{Note To Georg: #1})}

% fix some bad hyphenations
\hyphenation{par-a-mount}
\hyphenation{poly-pro-pyl-ene}
\hyphenation{his-to-grams}
\hyphenation{pres-ent-ly}
\hyphenation{dem-on-strate}
\thispagestyle{empty} \vspace*{-15mm}

\includegraphics[width=0.15\textwidth]{CMS-bw-logo}


\vspace*{-25mm}


{\large\sffamily
\hspace*{\fill}\parbox{55mm}%
{%
{\sffamily\bfseries {CMS-TDR-21-XXX}}                   \\[2mm]
{\sffamily\bfseries {\today}       \\[2mm]
}
}
}

\vspace{3cm}

\begin{center}

{\sffamily\bfseries \Huge The \phasetwo Upgrade}\\[15mm]
{\sffamily\bfseries \Huge of the CMS Beam Radiation}\\[15mm]
{\sffamily\bfseries \Huge Instrumentation and}\\[15mm]
{\sffamily\bfseries \Huge Luminosity Detectors}\\[15mm]
{\sffamily\bfseries \huge Technical Design Report}\\[30mm]
{\sffamily \huge CMS Collaboration}\\[10mm]
{\sffamily \large Draft Version}\\[10mm]
{\sffamily \large \today}\\[10mm]

\end{center}

\vspace*{\fill}
\clearpage
\section*{Editors}
A.~Dabrowski, D.~Stickland, G.~Auzinger, P.~Lujan
 
\section*{Chapter editors}
G.~Auzinger, I.~Azhgirey,  A.~Dabrowski, A.~Lokhovitskiy,  S.~Mallows, A.~B.~Meyer,  G.~P\'asztor, A.~Pozdnyakov,  D.~Stickland,  Z.~Xie


  
\vspace*{\fill}
\cleardoublepage
%\maketitle %maketitle comes after all the front information has been supplied
% >> Text
%%%%%%%%%%%%%%%%%%%%%%%%%%%%%%%%  Begin text %%%%%%%%%%%%%%%%%%%%%%%%%%%%%
\pagestyle{fancy}
\tableofcontents
%\cleardoublepage

%% **DO NOT REMOVE THE BIBLIOGRAPHY** which is located before the appendix.
%% You can take the text between here and the bibiliography as an example which you should replace with the actual text of your document.
%% If you include other TeX files, be sure to use "\input{filename}" rather than "\input filename".
%% The latter works for you, but our parser looks for the braces and will break when uploading the document.
%%%%%%%%%%%%%%%

% >> acknowledgments (for journal papers only)
% The latest version of the acknowledgments will be included from https://twiki.cern.ch/twiki/bin/viewauth/CMS/Internal/PubAcknow as of the date of submission. Modify to match either US or UK English spelling for centre/center, programme/program. For PRL use the short version, for JINST normally use the long version. All others take the middle length version other than exceptional cases.
%\begin{acknowledgments}
%We would like to thank all colleagues who contributed material (text, figures, and results) to this document. 
%\nts{we need something better here!}
%\end{acknowledgments}
%Part I
%\part{Project Overview \& Introduction}
%\chapter{The HL-LHC and the CMS \phasetwo Upgrade}
%\chapter{Physics Requirements for Luminosity Precision}
\nte{change chapter title as appropriate}
\nte{this is where the physics intro should go}
%\chapter{Radiation Simulation Deliverable}
\nte{Motivate the radsim deliverable for CMS. Describe what estimators CMS needs and why and the required precision. Describe the systematics and motivate the benchmarking. place the requirements on what the deliverable is.   E.g. 1 MeV n eq. Fluence estimates on the balconies for the electronics.  E.g Activation predictions.}
%

\chapter{Beam and Radiation Monitoring Strategy}
\nte{this chapter is the overview where the various BRIL systems, their purpose and deliverables are briefly described - detailed technical descriptions should follow in Part II of the document}

\section{Safety and Beam Abort}
\nte{Describe tolerances from the tracker for damage.  What conditions could approach these tolerances. Describe a LHC failure scenario that the BCML would need to detect. Describe UFO (example from Florian thesis that went to 98 percent of abort)}

\section{Beam Induced Background Measurement}
\nte{Introduce the LHC simulations and beam background sources. From simulations, summarise LHC vs HL-LHC expectations for both sources.
Monitoring purpose for CMS. Describe the sources. Beam gas, near CMS. Tracker fluence and safety of the tracker, contribution to slow down track reconstruction (tbc).  Vacuum quality of the first triplet itself should also be monitored, in addition to the beam gas signatures in the instrumentation / cms sub-detectors. Strategy to measure with TEPX\_D4R1 and … 
Higher radius beam background, interactions with the TCT $>$ 150m from the IP. distance beam gas. Strategy to measure with BHM and ME4 and … 
Any operational aspects that affect CMs, e.g. contaminating the trigger rate / quality, track reconstruction, mis ID of missing MET etc.}

\section{Beam Timing}

\section{Radiation Monitoring}
%\chapter{Luminosity Measurement Strategy}
\nte{again, this chapter should provide an overview of the luminosity deliverables, briefly describe calibration techniques, the foreseen instrumentation as a whole etc. - technical details for each system in Part II}

\section{Calibration Techniques}

\section{Luminosity Instrumentation}

\section{Luminosity Triggers and Special Requirements}


%\chapter{BRIL Data Acquisition Strategy and LHC Interfaces}
%Part II
%\part{Technical System Descriptions}
%\chapter{Safety and Beam Abort}
%\input{tex/Part2/Beam_Timing}
%\chapter{Neutron monitors based on gas-filled proportional counters}
%\chapter{BRIL Neutron Monitor 2 (e.g. Timepix3 with neutron conversion layers (tbc))}
%\chapter{BRIL Neutron Monitor 3 (e.g. \texorpdfstring{$^{6}$Li}{Li-6} Scintillator and SiPM (tbc))}

%%\chapter{Additional Radiation Monitors}

\section{PIN diode}
%\chapter{Hadron Forward Calorimeter Luminosity}
%\chapter{Tracker Endcap Pixel Detector (TEPX) for Luminosity Measurement}
%\chapter{TEPX Disk 4 Ring 1 for Luminosity and Beam Induced Background Measurement}
%\chapter{High-Precision Stand-Alone Luminometer - 1 (e.g. A Silicon-pad Detector) (tbc)}
%\chapter{High-Precision Stand-Alone Luminometer - 2 (e.g. A Quartz Fiber Luminometer) (tbc)}
\chapter{Outer Tracker Luminometer}

The CMS Phase II Outer Tracker  (OT) system will provide a source of high rate physics objects: L1 track stubs.
In its designed architecture these objects are sent to the L1 track reconstruction system at full 40 MHz frequency~\cite{CERN-LHCC-2017-009}.
Studies of CMS Phase II simulations show excellent linearity in the counting of these objects up to pile-up of 200.
These properties make this an excellent candidate for a precision luminometer.
In the following sections the detector layout, a preliminary design of the data acquisition system, and the expected perfomace are described.


\section{Detector Layout and track stub reconstruction}

The design of the Phase II OT is described in Chapter 3 of The Phase-2 Upgrade of the CMS Tracker TDR ~\cite{CERN-LHCC-2017-009} and shown below in Figure~\ref{fig:OT_layout}.
The layout consists of 6 barrel layers (3 TBPS + 3 TB2S) and 5 endcap (TEDD) disks.
The design of the CMS Phase II L1 trigger system requires the reconstruction of track stubs in the front-end of the OT in order to aid the track reconstruction for the L1 trigger.
For the luminosity measurement, CMS detector simulations have been studied up to the Phase II expected pile-up,
these studies described below show that the best precision can be obtained with the track stub counting from the barrel layer 6. 
The OT barrel layer 6 consists of 78 ladders on each side of the detector, with each ladder containing 12 sensor modules as shown in Figure~\ref{fig:OT_ladder_stub}.
The track stub reconstruction (right side of Figure ~\ref{fig:OT_ladder_stub}) is performed with firmware algorithms in the front-end electronics and transmitted to the back-end for all LHC bunch crossings.


\begin{figure}[hbtp]
\centering
\begin{subfigure}
\centering
\includegraphics[width=.9\linewidth]{tex/Part2/fig/OT/OT-longitudinal.png}
\end{subfigure}
\caption{
  Longitudinal view of the CMS Phase II Outer Tracker layout  with two regions in the barrel (TB2S, TBPS) and the endcap region (TEDD).
  For the luminosity measurement only the barrel outermost layer 6 in the TB2S part is used, this layer consists of 78 sensor ladders on each side of CMS.
}
\label{fig:OT_layout}
\end{figure}


\section{Data Acquisition}

A design of the OT DAQ including both the front-end and back-end systems is shown in Figure~\ref{fig:OT_DAQ}.
The back-end consists of Data, Trigger, and Control (DTC) boards which process the Trigger (TRIG) stream containting the track stub information. 
The list of track stubs reconstructed for each bunch crossing is transmitted in blocks of 8 bunch crossings to the DTC boards.
For the luminosity measurement, a specialized {\it BRIL Histogramming} firmware module has been developed which will count the number of track stubs in each bunch crossing and produce a histogram per orbit (3654 bunch crossings) at regular integration intervals of 1 second.

Each luminosity histogram will be configured to integrate track stubs from one 12 module ladder, the DTC boards process 3-6 ladders.
The number of necessary instances of  {\it BRIL Histogramming} modules will be installed in the DTC boards and readout via IPBus network protocol to the BRIL-DAQ system for further processing and storage.
Considering the rate of track stubs per ladder estimated with CMS simulations at pile-up of 200  (Figure~\ref{fig:OT_rates}),  the necessary memory per luminosity histogram is estimated using 32-bit memory words as follows: 3564 stub counting words (bins/orbit) + 9 words header + 2 words module mask + 192 error words (12 modules * 8 errors * 2 words/error) for a total of 3767 words per histogram.
In the case of 6 ladders per DTC this corresponds to a data readout rate of ~496 kbps.
For the total of 156 OT histograms, the total data rate received by the BRIL-DAQ is approximately 18.8 Mbps.


\clearpage

\begin{figure}[t]
\centering
\begin{subfigure}
  \centering
  \includegraphics[width=.48\linewidth]{tex/Part2/fig/OT/OT-ladder.png}
\end{subfigure}
\begin{subfigure}
  \centering
  \includegraphics[width=.48\linewidth]{tex/Part2/fig/OT/OT-stub.png}
\end{subfigure}
\caption{
  Layout of one OT sensor ladder with 12 modules (left) and diagram of the track stub reconstruction (right) using the two sensor layers in each module.  
}
\label{fig:OT_ladder_stub}
\end{figure}



\begin{figure}[hbtp]
\centering
\includegraphics[width=.85\linewidth]{tex/Part2/fig/OT/OT-DAQoverview.png}
\caption{
  Design of the Phase II Outer Tracker frontend (FE) and backend (BE) electronics~\cite{CERN-LHCC-2017-009}.  
}   
\label{fig:OT_DAQ}
\end{figure}


\clearpage

\section{Expected Performance}
%- Rates\\
%- linearity \\
%- statistical precision

The performance of the Phase II OT has been studied using detailed detector simulations for a wide range of pile-up scenarios. The samples have been privately produced using CMSSW version 11\_2\_0\_pre6 and a custom geometry that includes only the tracker volume with geometry versions 6.1.3 and 6.1.6 for the Inner and Outer Tracker, respectively.They also contain simulated effects from front-end electronics, specifically for the OT 2S modules the CBC hit detection mode is set to latched. Similar to what was done for TEPX simulations (see section \ref{sec:TEPX_sim}), the process that has been used to generate these samples is a single-neutrino event overlaid with a variable number of minimum-bias events to simulate pile-up. 

To test the linearity, the number of stubs is histogrammed per event for each layer of the OT. The mean of these distributions is then plotted as a function of pile-up, and a line is fitted to the lowest pile-up points (between 0 and 2) and then extrapolated to higher values, up to a pile-up of 200. Figure \ref{fig:OT_linearity} shows the linearity of stubs for all OT barrel layers. The deviation from a perfectly linear behaviour is presented in figure \ref{fig:OT_deviation} for all barrel layers (left) and for layer 6 only (right). It can be seen that layer 6 has the best linearity with a deviation of less than 1\% across the full pile-up range.

\begin{figure}[h!]
\centering
\includegraphics[width=.6\linewidth]{tex/Part2/fig/OT/OT-linearity.png}
\caption{
 Average number of track stubs per event as a function of pile-up determined from the CMS Phase II simulation showing a linear behaviour.
} 
\label{fig:OT_linearity}
\end{figure}

\begin{figure}[h!]
\centering
\includegraphics[width=.6\linewidth]{tex/Part2/fig/OT/OT-deviation.png}
\includegraphics[width=.6\linewidth]{tex/Part2/fig/OT/OT-deviation-L6.png}
\caption{
 Deviation from linearity for all barrel layers of the OT. \nts{PLACEHOLDER: The plan is to include in this figure two plots. On the left, the deviation from linearity for all barrel layers, and on the right, the deviation from linearity for L6 only including the 1\% line. To be done in CMS style.}
}
\label{fig:OT_deviation}
\end{figure}

Another important figure of merit for the performance of the OT as a luminometer is the statistical precision that can be achieved. Figure \ref{fig:OT_rates} shows the average number of stubs per event per ladder in layer 6, obtained from the simulations described above. The maximum total rate for the pile-up step of 200 is found to be of 902 stubs per event for Physics and of 2.255 for vdM conditions. With these counts, and considering a trigger readout rate of 40~MHz and an integration period of 1~s (30~s), a statistical precision of 0.03\% (0.115\%) per bunch-crossing is expected for Physics (vdM) conditions.   

\begin{figure}[h!]
\centering
\includegraphics[width=.6\linewidth]{tex/Part2/fig/OT/OT-Rates.pdf}
\caption{
  Average number of track stubs per event per ladder as a function of the ladder id.
  The rate is determined from the CMS Phase II simulation for a pile-up of 200. \nts{PLACEHOLDER: To be replaced with plot using CMS style.}
}
\label{fig:OT_rates}
\end{figure}





%\chapter{Luminosity Trigger Generation and Special Clocking Scheme}
\chapter{Muon Drift Tube Luminometer}

- Overview \\
- past experience \\
- motivation

\section{Detector Layout}
- Description of the DT system layout \\
- trigger primitive objects

\section{Muon Trigger Primitive Counting and Data Acquisition}
- DT backend \\
- trigger primitive histogramming \\
- data transfer

\section{Expected Performance}
- Rates \\ 
- linearity \\
- statistical precision 

\clearpage

\begin{figure}[hbtp]
\centering
\begin{subfigure}
  \centering
  \includegraphics[width=.9\linewidth]{tex/Part2/fig/DT/DT-longitudinal.png}
\end{subfigure}
\begin{subfigure}
  \centering
  \includegraphics[width=.6\linewidth]{tex/Part2/fig/DT/DT-transverse.png}
\end{subfigure}
\caption{Geometrical layout of the CMS Muon barrel detector system showing the Drift Tube (DT) chambers longitudinal view (top) and transverse view (bottom). The numbering scheme for the chambers is shown in within the figure.}
\label{fig:DT_layout}
\end{figure}


\begin{figure}[hbtp]
\centering
\includegraphics[width=.65\linewidth]{tex/Part2/fig/DT/DT-DAQoverview.png}
\caption{Design  of the Muon DAQ system architecture for Phase II showing the different layers where the L1 trigger objects are generated. In  this layout the DT trigger primitives are generated in the Layer 1 of the backend.}   
\label{fig:DT_DAQ}
\end{figure}


\begin{figure}[hbtp]
\centering
\includegraphics[width=.7\linewidth]{tex/Part2/fig/DT/DT-RatesExtrapolated.png}
\caption{DT L1 trigger primitive rates for the chambers in the wheel YB+2.} 
\label{fig:DT_rates}
\end{figure}


\begin{figure}[hbtp]
\centering
\includegraphics[width=.7\linewidth]{tex/Part2/fig/DT/DT-Linearity.png}
\caption{DT L1 trigger primitive rates for one chamber  as a function of instantenous luminosity measured by the HF luminometer in Run 2 data.} 
\label{fig:DT_linearity}
\end{figure}

%\chapter{Trigger Scouting System Luminosity}


\section{BIB monitoring with EMTF++}

Another potential use of the scouting data will be for monitoring of high radius beam-induced backgrounds.
As described in section~\ref{sec:Chapter5:BIBTheory}, these particles may arise from the interaction of the beam with the collimators or beam gas far from CMS and arrive with trajectories aproximately parallel to the beam at high radius.
For Phase-2, the upgraded Endcap Muon Track Finder algorithm (EMTF++) is designed to reconstruct transversly displaced muons \cite{CERN-LHCC-2020-004}.
Displaced muons reconstruced by the EMTF++ would be sent the GMT together with the normal endcap muon list.
These L1 muon candidates may be selected from the GMT muons and  counted similarly to the BMTF muons.
This measurement will provide an additional beam quality flag in real time, redundant to the BHM system.
In the studies presented in \cite{CERN-LHCC-2020-004}, the efficiency for detecting displaced muons with the EMTF++ was approxmately 50\% and constant as a function of the transerverse displament ($d_0$), however the muon sample used for those studies only covered $d_0<$1.2 m.
At the time of writing, further studies are underway including beam halo simulation to understand the efficiency of the reconstruction algorithm for tracks with angles parallel to the beam and at higher radius.


%%Displaced muon track reco at EMTF
%%Muon list sent to GMT
%%GMT List sent to Scouting
%%Select tags and histogram at Scouting
%%Send to BRILDAQ along with BMTF, EMTF, OMTF
%%
%%TDR-021 Page 97: displaced muons
%%- d0 < 120 cm
%%- 1.2 < |eta| <2.4
%%- |z0| < 30 cm
%%
%%BHM from BRIL TDR:
%%- r=180 - 190 cm  
%%- distancefrom the IP of z = 2000 - 2030 cm

\begin{figure}[hbtp]
\centering
\includegraphics[width=.7\linewidth]{tex/Part2/fig/Phase2TriggerProject-TDR021-Page8.png}
\caption{
Design of the Phase-2 Trigger Project [TDR-021].
}
\label{fig:Phase2TriggerProject}
\end{figure}

\begin{figure}[hbtp]
\centering
\includegraphics[width=.97\linewidth]{tex/Part2/fig/EMTFpp_eff_vs_eta_d0.png}
\caption{
EMTF++ displaced muon efficiency [TDR-021].
}
\label{fig:Phase2TriggerProject}
\end{figure}

\begin{figure}[hbtp]
\centering
\includegraphics[width=.7\linewidth]{tex/Part2/fig/DT-longitudinal.png}
\caption{
Muon system layout.
}
\label{fig:MuonSystem}
\end{figure}






%\input{tex/Part2/BHM}
%\chapter{Additional Radiation and Neutron Monitors}

\section{RAMSES}

\section{LHC RADMON}

\section{RadFET}

\section{PIN diode}
%\chapter{BRIL Data Acquisition System}

%Part III
%\part{Project Organisation, Responsibilities, Planning \& Cost}
%\input{tex/Part3/Project_Organisation}
%\chapter{Institutional Interests and Responsibilities - Task Sharing}
%\chapter{Project Schedule and Milestones}
%\input{tex/Part3/Cost_Estimate}
%\input{tex/Part3/Risk_Analysis}

%% **DO NOT REMOVE BIBLIOGRAPHY**
\bibliography{briltdr2021}   % will be created by the tdr script.
%% examples of appendices.
%\clearpage
%\appendix
%\section{Appendix name}
%%% DO NOT ADD \end{document}!

