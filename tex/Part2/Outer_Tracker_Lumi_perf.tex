\section{Expected Performance}
%- Rates\\
%- linearity \\
%- statistical precision

The performance of the Phase II OT has been studied using detailed detector simulations for a wide range of pile-up scenarios. The samples have been privately produced using CMSSW version 11\_2\_0\_pre6 and a custom geometry that includes only the tracker volume with geometry versions 6.1.3 and 6.1.6 for the Inner and Outer Tracker, respectively.They also contain simulated effects from front-end electronics, specifically for the OT 2S modules the CBC hit detection mode is set to latched. Similar to what was done for TEPX simulations (see section \ref{sec:TEPX_sim}), the process that has been used to generate these samples is a single-neutrino event overlaid with a variable number of minimum-bias events to simulate pile-up. 

To test the linearity, the number of stubs is histogrammed per event for each layer of the OT. The mean of these distributions is then plotted as a function of pile-up, and a line is fitted to the lowest pile-up points (between 0 and 2) and then extrapolated to higher values, up to a pile-up of 200. Figure \ref{fig:OT_linearity} shows the linearity of stubs for all OT barrel layers. The deviation from a perfectly linear behaviour is presented in figure \ref{fig:OT_deviation} for all barrel layers (left) and for layer 6 only (right). It can be seen that layer 6 has the best linearity with a deviation of less than 1\% across the full pile-up range.

\begin{figure}[h!]
\centering
\includegraphics[width=.6\linewidth]{tex/Part2/fig/OT/OT-linearity.png}
\caption{
 Average number of track stubs per event as a function of pile-up determined from the CMS Phase II simulation showing a linear behaviour.
} 
\label{fig:OT_linearity}
\end{figure}

\begin{figure}[h!]
\centering
\includegraphics[width=.6\linewidth]{tex/Part2/fig/OT/OT-deviation.png}
\includegraphics[width=.6\linewidth]{tex/Part2/fig/OT/OT-deviation-L6.png}
\caption{
 Deviation from linearity for all barrel layers of the OT. \nts{PLACEHOLDER: The plan is to include in this figure two plots. On the left, the deviation from linearity for all barrel layers, and on the right, the deviation from linearity for L6 only including the 1\% line. To be done in CMS style.}
}
\label{fig:OT_deviation}
\end{figure}

Another important figure of merit for the performance of the OT as a luminometer is the statistical precision that can be achieved. Figure \ref{fig:OT_rates} shows the average number of stubs per event per ladder in layer 6, obtained from the simulations described above. The maximum total rate for the pile-up step of 200 is found to be of 902 stubs per event for Physics and of 2.255 for vdM conditions. With these counts, and considering a trigger readout rate of 40~MHz and an integration period of 1~s (30~s), a statistical precision of 0.03\% (0.115\%) per bunch-crossing is expected for Physics (vdM) conditions.   

\begin{figure}[h!]
\centering
\includegraphics[width=.6\linewidth]{tex/Part2/fig/OT/OT-Rates.pdf}
\caption{
  Average number of track stubs per event per ladder as a function of the ladder id.
  The rate is determined from the CMS Phase II simulation for a pile-up of 200. \nts{PLACEHOLDER: To be replaced with plot using CMS style.}
}
\label{fig:OT_rates}
\end{figure}

